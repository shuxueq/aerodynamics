% !TEX root =../mechanics.tex
%high-speed_presure_flow.tex
%高速可压流动

\chapter{高速可压流动}
\section{热力学基础}

\subsection{内能(internal energy)}
对于遵守$P=\rho RT$的气体称为{\bfseries 热完全气体}
\index{热完全气体},这种理想化的
完全气体,内能只计微观运动的平均动能,因此内能只与
绝对温度$T$有关,单位完全气体的内能记为$e$,单位
是$\mathrm{J /Kg}$.于是有
\[
	e=e(T)
\]
内能是与变化过程无关的参数.

\subsection{焓(enthalpy)}
焓表示单位质量气体的内能和压力能之和,用$h$表示
\[
	h=e+\frac{p}{\rho}
\]
其中$\frac{P}{\rho}$表示单位质量气体的压力能,对于
完全气体,焓也取决于温度,所以焓也是一个状态参数.

\subsection{热力学第一定律和比热}
外界传递给一个封闭物质系统的热量等于系统内能的增量
和系统对外界所做的机械功的总和,这就是热力学第一定律.

对于单位质量的气体有

\begin{equation}
	\delta q =\mathrm{d}e +P \mathrm{d} \frac{1}{\rho}
	\label{eq:4}
\end{equation}
$\frac{1}{\rho}$是单位质量气体所占的体积,称为比容.
又
\[
	\mathrm{d}h =\mathrm{d} e+\mathrm{Pd}\frac{1}{\rho}+\frac{1}{\rho}\mathrm{d}P
\]
于是
\[
	\delta q =\mathrm{d} h-\frac{1}{\rho}\mathrm{d}P
\]
若对于等压过程 $\mathrm{d}P=0$,焓增量$\mathrm{d}h$就等于
此过程张吸收的热量$\delta q$.若一个系统
由于加给一微小的热量$\delta q$ 而升高了温度
$\mathrm{d} T$,定义比值
\[
	\delta q /\mathrm{d}T
\]
为热容,单位J/K.单位质量上的热容为比热容,
简称比热,单位 J/(Kg$\cdot$T).在定容情况下,
$\mathrm{d} \dfrac{1}{\rho}=0$,定压情况下,
$\mathrm{d}P=0$,于是
\begin{equation}
	\begin{split}
		C_v & =\left(\frac{\delta q}{\mathrm{d} T}\right)_{\rho=const}=\frac{\mathrm{d}e}{\mathrm{d}T} \\
		C_P & =\left(\frac{\delta q}{\mathrm{d} T}\right)_{P=const}=\frac{\mathrm{d}h}{\mathrm{d}T}
		\label{eq:5 }
	\end{split}
\end{equation}
其中,$C_v$为比定容热容,$C_P$为比定压热容.取$T=0$时,
$e=h=0$,于是
\begin{align}
	e & =C_vT  \\
	h & =C_P T
\end{align}
将比定容热容和比定压热容之比称为
比热,用$\gamma$表示,即
\[
	\gamma=\frac{C_P}{C_v}
\]
对于空气,$\gamma=1.4$.又
\[
	h=e+\frac{P}{\rho}
\]
有
\[
	C_P T=C_v T+\frac{P}{\rho}
\]
于是
\[
	C_P=C_v+\frac{P}{\rho T}=C_v+R
\]
因此,
\[
	h=C_P T=\frac{\gamma R}{\gamma-1}T=\frac{\gamma RT}{\gamma-1}=\frac{\gamma}{\gamma-1}\frac{P}{\rho}
\]
其中$C_v=\frac{R}{\gamma-1 }$,$C_P=\frac{\gamma R}{\gamma-1}$.

\begin{notice}
  {\bfseries 热完全气体(calorically perfect gas)}是
	指$C_v$和$C_P$都是常数的气体.
\end{notice}


\subsection{热力学第二定律和熵(entropy)}
热力学第二定律指明能量相互转化是有条件的,有方向的,即从
一个方向的变化过程可以实现,而逆向的变化过程不能实现或者
只能有条件的实现.

定义单位质量的气体的熵增量$\mathrm{d} s$为单位质量气体
的热量增量与绝对温度的比值,即
\[
	\mathrm{d} s=\frac{\delta q}{T}
\]
又
\[
	\delta q=\mathrm{d} e +P \mathrm{d} \frac{1}{\rho}
\]
于是,
\begin{equation}
	\begin{split}
		\begin{WithArrows}[code-before = \color{blue}]
			\mathrm{d} s &=\frac{\delta q}{T}=\frac{1}{T}(\mathrm{d} e +P \mathrm{d}  \frac{1}{\rho}) \\
			&=\frac{1}{T}(\mathrm{d} C_vT+P \mathrm{d} \frac{1}{\rho})
			\Arrow{$P=\rho RT$} \\
			&=\frac{1}{T}C_v \mathrm{d}T+\frac{P}{T} \mathrm{d} \frac{1}{\rho} \\
			&=\frac{1}{T}C_v \mathrm{d} T+\rho R \mathrm{d} \frac{1}{\rho} \\
			&=\mathrm{d}\left(C_v \ln T+R\ln \frac{1}{\rho}\right)
		\end{WithArrows}
	\end{split}
\end{equation}
因此,熵是一个状态函数,当系统由状态1($s_1$,$P_1$,$\rho_1$,$T_1$)变化为状态2$(s_2$,$P_2$,$\rho_2$,$T_2)$时,
过程的熵增量为
\begin{equation}
	\begin{split}
		\Delta s=s_2-s_1 & =C_v\ln T_2+R\ln \frac{1}{\rho_2}-\left(C_v\ln T_1+R\ln \frac{1}{\rho_1}\right) \\
		                 & =C_v \ln \frac{T_2}{T_1}+R\ln \frac{\rho_1}{\rho_2}
	\end{split}
	\label{eq:6}
\end{equation}
由$C_P=C_v+R$,$\gamma =\frac{C_P}{C_v}$,有
\[
	C_v=\frac{R}{\gamma-1}
\]
于是
\begin{equation}
	\begin{split}
		\Delta s & =C_v\ln \frac{T_2}{T_1}+C_v(\gamma-1)\ln \frac{\rho_1}{\rho_2}                                        \\
		         & =C_v \ln\left[\frac{T_2}{T_1}\cdot\left(\frac{\rho_1}{\rho_2}\right)^{\gamma-1}\right]                \\
		         & =C_v \ln \left[\frac{T_2}{T_1}\left(\frac{\rho_1}{\rho_ 2}\right)^\gamma \frac{\rho_2}{\rho_1}\right] \\
		         & =C_v \ln \left[\frac{P_2}{P_1}\left(\frac{\rho_1}{\rho_2}\right)^\gamma\right]
	\end{split}
\end{equation}
上式对于不可逆状态依然适用,前提是热力学系统处于平衡状态.

热力学第二定律指出,在绝热变化过程的孤立系统中,如果过程
可逆,则熵值保持不变.如果过程不可逆,则熵必增加.

一般来说,在流场的大部分区域,速度梯度和温度梯度并不是
很大,流动过程可以近似看成是绝热的,熵增等于0,这样的流动称为
等熵流(isentropical flow).沿一条流线熵值保持不变的情况称为沿流线等熵.
全流场熵值不变的流动称为均匀等熵流动.

对于等熵流即绝热可逆流动有
\[
	\frac{P_2}{P_1}\left(\frac{\rho_1}{\rho_2}\right)^\gamma=1\Rightarrow
	\frac{P_2}{\rho_2^\gamma}=\frac{P_1}{\rho_1^\gamma}=const
\]

\begin{notice}
等熵流动就是绝热可逆流动.
\end{notice}


\section{一维等熵绝热流}
气体参数发生在微小变化的扰动称为小扰动.小扰动的传播速度只
取决于气体的性质及状态参数,而与何种扰源和其成因无关. 小扰动
在气体中的传播速度习惯上称为{\bfseries 声速(speed of sound)}.

考虑声波以速度$a$在空气中传播,声波向左传播并进入参数
分别是$P$,$\rho$,$T$的静止气体,在声波之后,气体参数
变成了$P+\mathrm{d}P $,$\rho+\mathrm{d}\rho$,$T+\mathrm{d}T $.
选择声波前后的气体作为控制体{\color{red}如图},由连续方程有
\[
  \rho a=(\rho+\mathrm{d}\rho)(a+\mathrm{d}a)
\]
展开并略去高阶项有
\[
  \rho a=\rho a+a \mathrm{d}\rho+\rho \mathrm{d}a 
\]
解得
\[
  a=-\rho \frac{\mathrm{d}a }{\mathrm{d}\rho}
\]
再列动量方程有
\[
  P+\rho a^2=(P+\mathrm{d}P)+(\rho+\mathrm{d}\rho)(a+\mathrm{d}a)^2
\]
展开并略去比二阶高的小量有
\[
  \mathrm{d}P=-2\rho a \mathrm{d}a-a^2 \mathrm{d}\rho
\]
解得
\[
  \mathrm{d}a =\frac{\mathrm{d}P+a^2 \mathrm{d}\rho}{-2a \rho}
\]
代入到
\[
  a=-\rho \frac{\mathrm{d}a}{\mathrm{d}\rho} 
\]
得到
\[
  a^2=\frac{\mathrm{d}P}{\mathrm{d}\rho}
\]

上述讨论中,流体流过声波是等熵的,因此压强对于密度的变化
就是等熵的,可以将方程
\[
  a^2=\frac{\mathrm{d}P }{\mathrm{d}\rho}
\]
重写成
\[
  a^2=\left(\frac{\partial P }{\partial \rho}\right)_s
\]

假设气体是量热完全气体,满足方程
\[
  \frac{{P_1}}{P_2}=\left(\frac{\rho_1}{\rho_2}\right)^\gamma
\]
又
\[
  \frac{P}{\rho^\gamma}=const=C 
\]
或者写成
\[
  P=C \rho^\gamma
\]
于是
\[
  \left(\frac{\partial P }{\partial \rho}\right)_s=C \gamma \rho^{\gamma-1}
\]
再将
\[
  C=\frac{P}{\rho^\gamma}
\]
代入,得到
\[
  \left(\frac{\partial P }{\partial \rho}\right)_s=\frac{P}{\rho^\gamma}\gamma \rho^{\gamma-1}
  =\gamma \frac{P}{\rho}=\gamma RT 
\]

\begin{notice}
$a^2=\gamma R T $是量热完全气体等熵流动的计算公式,说明声速
仅仅只是温度$T$的函数。
\end{notice}

上式中定义的变量 $ a $ 就是小扰动传播的速度——声速.

\subsection{能量方程}

对于一维等熵绝热流,能量方程可由欧拉方程并利用等熵关系式沿
流线积分求出.

由一维伯努利方程,忽略重力势
\[
	\frac{1}{2 }\mathbf{V}^2+\int \mathrm{d}\frac{P}{\rho}=C
\]
因为沿流线等熵,由等熵关系
\[
	\frac{P}{\rho^\gamma}=C
\]
有
\[
	\frac{1}{2 }V^2+\int \mathrm{d}\frac{P}{\rho}=\frac{1}{2 }V^2+\frac{\gamma}{\gamma-1 }\frac{P}{\rho}
\]
又
\[
	h=\frac{\gamma}{\gamma-1 }\frac{P}{\rho}
\]
于是
\[
	\frac{1}{2 }V^2+h=C
\]
对于定常绝热流,上述方程不论是否等熵,在形式上都成立,也就是绝热流
动中粘性摩擦并不改变动能和焓的总和,而是将一部分动能转化为
焓. 上式还可以改写成
\begin{equation}
	\begin{split}
		\frac{1}{2 }V^2+\frac{\gamma}{\gamma-1}RT & =
		\frac{1}{2 }V^2+\frac{\gamma}{\gamma-1}\frac{P}{\rho}                                \\
		                                          & =\frac{1}{2 }V^2+\frac{a^2}{\gamma-1} =C
	\end{split}
	\label{eq:8 }
\end{equation}
从中可以看出,对于定常绝热流,当沿流线速度变大的时候
温度,声速,焓都要减小,但是动能和焓的总和不变.

\subsection{参数间的基本关系式}
\begin{enumerate}
	\item 驻点 \\
	      指流速沿流线等熵地降为0的那一点,该点的参数称为
	      驻点参数,或者总参数.对应这个状态的焓,温度,压强,
	      密度分别称为总焓,总温,总压,总密度。用$h_0$,$T_0$,$P_0$,$\rho_0$
        表示,驻点状态的声速用$a_0$表示.
	      \begin{notice}
		      总温和总焓是指速度绝热地降为0的那点的温度和焓值.总密度
		      和总压
		      则是指速度等熵地降为0的那点的密度和压强.
	      \end{notice}
	      于是上述能量方程就可以改写为
	      \begin{empheq}[box=\widefbox]{align}
		      \frac{1}{2 }V^2+\frac{\gamma}{\gamma-1}\frac{P}{\rho} & =\frac{\gamma}{\gamma-1}\frac{P_0}{\rho_0} \\
		      \frac{1}{2 }V^2+h                                     & =h_0                                       \\
		      \frac{1}{2 }V^2+\frac{a^2}{\gamma-1}                  & =a_0^2                                     \\
		      \frac{1}{2 }V^2 +\frac{\gamma RT }{\gamma-1 }         & =\frac{\gamma}{\gamma-1 }RT_0
	      \end{empheq}
	      显然对于一维定常绝热等熵流,$h_0$,$P_0$,$T_0$,$\rho_0$和$a_0$
	      沿同一条流线恒等于常数而不发生改变,和驻点参数对应的是流动过程中任意一点
	      处的当地流动参数$h$,$P$,$T$,$\rho$等称为静参数,即静焓,静压,静温,静密度.
	      因此上述方程也可以写成
	      \[
		      \frac{1}{2 }\frac{V^2}{\gamma RT}+\frac{1}{\gamma-1 }=\frac{1}{\gamma-1 }\frac{T_0}{T }
	      \]
	      也就是
	      \begin{gather*}
		      1+\frac{\gamma-1 }{2 }\frac{V^2}{a^2}=1+\frac{\gamma-1 }{2 }M^2=\frac{T_0}{T }  \\
		      \frac{1}{2 }\frac{V^2}{h^2}+1 =\frac{h_0}{h },(h=\frac{\gamma}{\gamma-1 }RT)                   \\
		      \Rightarrow \frac{\gamma-1 }{2 }\frac{V^2}{\gamma RT }+1
		      =\frac{h_0}{h }                                                  \\
		      \Rightarrow \frac{\gamma-1 }{2 }\frac{V^2}{a^2}+1
		      =\frac{h_0}{h }
	      \end{gather*}
	      也就是
	      \begin{empheq}[box=\bluebox]{equation}
		      \frac{h_0}{h }=1+\frac{\gamma-1 }{2 }M^2
	      \end{empheq}
	      利用等熵关系$\frac{P}{\rho^\gamma}=C $可得
	      \[
		      \frac{P}{\rho}=C \rho^{\gamma-1}
	      \]
	      于是有
	      \begin{align*}
		      \frac{1}{2 }V^2+\frac{C \gamma}{\gamma-1 }\rho^{\gamma-1 }             & =\frac{C \gamma}{\gamma-1 }\rho_0^{\gamma-1 }               \\
		      \Rightarrow \frac{1}{2 \rho^{\gamma-1}}V^2 +\frac{C \gamma}{\gamma-1 } & =\frac{C \gamma}{\gamma-1 }(\frac{\rho_0}{\rho})^{\gamma-1}
	      \end{align*}
	      又$\rho^{\gamma-1}=\frac{P}{C \rho}=\frac{RT }{C }$
	      \[
		      \Rightarrow \frac{C}{2RT }V^2+\frac{C \gamma}{\gamma-1}=\frac{C \gamma }{\gamma -1}(\frac{\rho_0}{\rho}^{\gamma-1})
	      \]
	      也就是
	      \[
		      \frac{1}{2\gamma RT }V^2+\frac{1}{\gamma-1}=\frac{1}{\gamma-1 }(\frac{\rho_0}{\rho})^{\gamma-1}
	      \]
	      也就是
	      \begin{empheq}[box=\bluebox]{equation}
		      \frac{\rho_0}{\rho}=\left(1+\frac{\gamma-1 }{2 }M^2\right)^{\frac{1}{\gamma-1}}
		      \label{eq:9}
	      \end{empheq}
	      再利用等熵关系有
	      \begin{equation}
		      \frac{\rho_0}{\rho}=\left(\frac{P_0}{P }\right)^{\frac{1}{\gamma}}
		      \label{eq:10}
	      \end{equation}
	      于是
	      \begin{empheq}[box=\bluebox]{equation}
		      \begin{split}
			      \frac{P_0}{P } & =\left[\left(1+\frac{\gamma-1 }{2 }M^2\right)^{\frac{1}{\gamma-1 }}\right]^\gamma \\
			                     & =\left(1+\frac{\gamma-1 }{2 }M^2\right)^{\frac{\gamma}{\gamma-1}}
		      \end{split}
		      \label{eq:11}
	      \end{empheq}
	      其中
	      \begin{empheq}[box=\bluebox]{equation*}
		      M=\frac{V}{a}
	      \end{empheq}
	      称为流动马赫数,是反映流体压缩性能大小的相似准则,是当地流动速度和
	      当地声速的比值.
        \begin{notice}
        当讨论飞机或者其他飞行物体时,马赫数就是这个物体的速度除以自由来流
        的声速,而不是流过该物体的流体的声速.(自由来流就是远前方的气流,流过
        该物体的流体的温度和远前方气流的速度并不相同,一般要高些)

        马赫数是流
        场的当地性质,流场每个点的马赫数都可能不同.

        马赫数还反映了流体内能和动能的比值,即
        \[
          \frac{\frac{V^2}{2}}{e}=\frac{\gamma(\gamma-1)}{2}M^2
        \]
        \end{notice}
        
	\item 临界状态 \\
	      临界状态是指当地流速等于当地声速的状态,也称为临界点,用下标* 表示,显然
	      在临界状态,当地马赫数$M=1 $,式
	      \[
		      \frac{V^2}{2 }+\frac{a^2}{\gamma-1}=\frac{a_0^2}{\gamma-1}
	      \]
	      可以写成
	      \[
		      \frac{V_*^2}{2 }+\frac{a_*^2}{\gamma-1 }=\frac{a_0^2}{\gamma-1 }
	      \]
	      又
	      \[
		      V_*=a_*
	      \]
	      于是
	      \begin{empheq}[box=\bluebox]{equation*}
		      a_*^2=\frac{2}{\gamma+1 }a_0^2
	      \end{empheq}
	      式
	      \[
		      \frac{T_0}{T }=1+\frac{\gamma-1 }{2 }M^2
	      \]
	      在临界状态下可以改写成
	      \[
		      \frac{T_0}{T_* }=1+\frac{\gamma-1 }{2 }=\frac{\gamma+1 }{2 }
	      \]
	      定义速度系数$\lambda$为
	      \[
		      \lambda =\frac{V}{a_* }
	      \]
	      由于
	      \begin{equation}
		      \begin{split}
			      M^2=\frac{V^2}{a^2} & =\frac{V^2}{a_*^2}
			      \frac{a_*^2}{a_0^2} \frac{a_0^2}{a^2}                                           \\
			                          & =\lambda^2 \frac{2}{\gamma+1} \frac{T_0}{T }
			      ,(a=\sqrt{\gamma RT } )                                                         \\
			                          & =\lambda^2 \frac{2}{\gamma+1 }(1+\frac{\gamma-1 }{2 }M^2)
		      \end{split}
		      \label{eq:12}
	      \end{equation}
	      于是
	      \[
		      M^2=\frac{\frac{2}{\gamma+1 }\lambda^2}{1-\frac{\gamma-1 }{\gamma+1 }\lambda^2}
	      \]
	      从式
	      \[
		      \frac{T_0}{T }=1+\frac{\gamma-1 }{2 }M^2
	      \]
	      中可以看出,当$T\rightarrow 0 $时,$M\rightarrow \infty$,
	      而$\lambda$趋于有限值,即
	      \[
		      \lambda_{\max}=\sqrt{\frac{\gamma+1 }{\gamma-1 }}
	      \]
	      使用速度系数$\lambda$可得
	      \[
		      \frac{T_0}{T }=1+\frac{\gamma-1 }{2 }M^2=1-\frac{\gamma-1 }{\gamma+1 }\lambda^2
	      \]
	      类似地
	      \begin{equation}
		      \begin{split}
			      \frac{P}{P_0}        & =
			      \left(1-\frac{\gamma-1 }{\gamma+1 }\lambda^2\right)^{\frac{\gamma}{\gamma-1}} \\
			      \frac{\rho}{\rho_0 } & =
			      \left(1-\frac{\gamma-1 }{\gamma+1 }\lambda^2\right)^{
			      \frac{1}{\gamma-1}
			      }
		      \end{split}
		      \label{eq:13}
	      \end{equation}
	      临界状态下,临界参数和总参数的对应关系如下
	      \begin{empheq}[box=\widefbox]{align}
		      \frac{T_*}{T_0}         & =1-\frac{\gamma-1}{\gamma+1 }                                         \\
		      \frac{P_* }{P_0}        & =\left(1-\frac{\gamma-1 }{\gamma+1 }\right)^{\frac{\gamma}{\gamma-1}} \\
		      \frac{\rho_* }{\rho_0 } & =\left(1-\frac{\gamma-1 }{\gamma+1 }\right)^{\frac{1}{\gamma-1 }}
	      \end{empheq}
	      从式
	      \[
		      \frac{1}{2 }V^2+\frac{\gamma}{\gamma-1 }RT=\frac{\gamma}{\gamma-1 }RT_{0}
	      \]
	      可以看出,当$T=0 $时速度达到最大,即
	      \[
		      V_{\max}=\sqrt{\frac{2\gamma}{\gamma-1 }RT_0}=\sqrt{\frac{2a_0^2}{\gamma-1 }}
	      \]
	      这种状态是一种假想状态,意味着给定一个总温后所能达到的最大流动
	      速度,是一个上限值,即流体中的全部焓转换成了动能.

	      若流动过程绝热,则总温不变,$T_{01}=T_{02}$.对于等熵绝热流
	      总压也保持不变,$P_{01}=P_{02}$.而对于一个绝热过程,如果
	      变化过程中有摩擦等损失存在,则该过程是不可逆过程,熵必有所增加
	      ,必然表现为$P_{01}>P_{02}$,即总压有所损失,记$\sigma=\frac{P_{02}}{P_{01}}$
	      为总压损失比.
\end{enumerate}

\begin{notice}
	关于总参数和静参数的理解

	如果飞机飞行高度在10000英尺,速度为0.8马赫.那么
	10000英尺高度的压强就是静参数,用皮托管测到的压强
	就是总参数(皮托管测量到的压强是速度为零的压强).
	将飞机静止来看,远前方的气流就是运动的,就是静参数.
\end{notice}

\section{马赫波和膨胀波}
亚声速流场中的小扰动可以遍及全场,在声速和超声速流动中小扰动不会干扰到干扰源
的上游.

超声速气流受到小扰动而使气流方向发生微小变化,扰动的界面就是马赫波.
设超声速定常直匀流沿壁面流动,若壁面外折\marginpar{解释壁面内折和
	外折的内容(插图)}伴随着流速增大,压强,密度,温度均减小,此时气流发生膨胀
,此时的马赫波称为膨胀马赫波. 壁面内折,则伴随着流速减小,压强,密度,温度
均增大,气流发生压缩,称为压缩马赫波. 气流通过马赫波后壁面上的压强系数
\[
	C_p=\frac{(P+\mathrm{d}P)-P}{\frac{1}{2 }\rho V^2}=
	\frac{\mathrm{d}P }{\frac{1}{2 }\frac{P}{RT }M^2a^2 }=-\frac{2}{\sqrt{M^2-1} }\mathrm{d}\theta
\]

对于大角度超声速气流发生折转的情况,流动参数与方向偏角之间的关系通过
膨胀波和激波建立.

膨胀波是超声速气流的基本变化之一,是一种压强下降,密度下降,而流速上升
的过程. 由于气流经过每一道膨胀马赫波气流参数只发生微小变化,因此穿过
整个扇形膨胀波时气流参数必是连续变化的,这种连续变化过程必是等熵的.
所以气流经过膨胀波是可逆等熵过程,对于一定的来流条件,波后气流只取决于
总的外折角$\theta$,对
\[
	V=aM
\]
取全微分,有
\[
	\mathrm{d}V=M \mathrm{d}a+a \mathrm{d}M
\]
也就是
\[
	\frac{\mathrm{d}V }{aM }=\frac{\mathrm{d}a }{a }+\frac{\mathrm{d}M }{M }\Rightarrow
	\frac{\mathrm{d}V }{V }=\frac{\mathrm{d}a }{a }+\frac{\mathrm{d}M }{M }
\]
而
\[
	\frac{\mathrm{d}V }{V }=\frac{\mathrm{d}\sqrt{\gamma RT } }{\sqrt{\gamma RT } }=
	\frac{\mathrm{d}M }{M }+\frac{\mathrm{d} T }{2T },(a=\sqrt{\gamma RT } )
\]
又
\[
	\frac{\mathrm{d} M }{M }=\frac{1+\frac{\gamma-1 }{2 }M^2}{\sqrt{M^2-1 } }\mathrm{d}\theta
\]
当壁面外折角由$0$增大到$\theta $时,马赫数由$M_1$增大到$M_2$有
\[
	\theta=\int_{M_1}^{M_2}\frac{\sqrt{M^2-1} }{(1+\frac{\gamma-1 }{2 }M^2)M}\mathrm{d}M
\]
积分得到(过程参见\ref{马赫波积分})
\begin{empheq}[box=\widefbox]{align*}
	\theta  =&\left[\sqrt{\frac{\gamma+1 }{\gamma-1 }} \arctan\sqrt{\frac{\gamma-1 }{\gamma+1 }(M_2^2-1)} -\arctan\sqrt{M_2^2-1} \right] \\
	-&\left[\sqrt{\frac{\gamma+1 }{\gamma-1 }} \arctan\sqrt{\frac{\gamma-1 }{\gamma+1 }(M_1^2-1)} -\arctan\sqrt{M_1^2-1} \right]
\end{empheq}
这样给定$M_1$和$\theta$后就可以求出$M_2$,在膨胀过程中总温,总压,总密度均不变,于是
便可以求出$\frac{P_2}{P_1}$,$\frac{T_2}{T_1}$,$\frac{\rho_2}{\rho_1}$.
若指定气流时从$M_1=1 $的声速流开始膨胀的,那么达到某个大于1的马赫数$M$
的外折角$\theta_* $
\begin{empheq}[box=\bluebox]{equation*}
	\theta_*=\sqrt{\frac{\gamma+1 }{\gamma-1 }} \arctan\sqrt{\frac{\gamma-1 }{\gamma+1 }(M_2^2-1)}
	-\arctan\sqrt{M_2-1}
\end{empheq}
于是
\[
	\theta_{*\max}=\lim_{M\rightarrow \infty}\theta_*=\left(\sqrt{\frac{\gamma+1 }{\gamma-1 }} -1\right)\frac{\pi}{2 }
\]
对于空气,$\theta_{*\max}=130.45^\circ$,此时已经膨胀到温度,压强,密度均降为
$0$的真空状态.

实际上,根据能量方程, 膨胀过程实际上时气体的焓值转换为动能的过程,真空状态
其焓值已经耗尽,因此不能再进行膨胀了.
\begin{notice}
    计算马赫数$M_1>1$的气流绕一个外折角$\nu$膨胀后的
    马赫数$M_2$.可以这样认为,一个$M=1$的气流先绕过一
    个外折角$\nu_1$膨胀后达到马赫数为$M_1$,然后在
    绕过外折角$\nu$膨胀后达到马赫数$M_2$,相当于$M=1$
    的气流绕过外折角为$\nu_2=\nu_1+\nu$膨胀后达到马赫数
    $M_2$.所以可以根据$M_1$先查普朗特---迈耶函数表,得到
    $\nu_1$,然后计算$\nu_2$,在根据$\nu_2$查普朗特---
    迈耶函数表得到$M_2$.
\end{notice}
\begin{example}
	高速导弹在滞止点的温度和压强分别是518.8K和7.8atm,计算这一点
	的密度.

	由完全气体状态方程
	\[
		P=\rho R T
	\]
	得到
	\[
		\rho_0 =\frac{P_0}{R T_0 }=\frac{7.8\times 1.01\times 10^5 }
		{518.8\times 287.05}=5.29 \text{Kg/m}^3
	\]
\end{example}

\begin{example}
	激波前气体温度和压强分别是288K和1atm;激波后气体温度和压强分别是
	690K和8.656atm.计算经过激波前后焓,熵,内能的变化.

	\[
		C_P=\frac{\gamma R}{\gamma-1}=\frac{1.4\times 287.05}{1.4-1}=1004.675
	\]
	\[
		C_v=\frac{R}{\gamma-1}=\frac{287.05}{1.4-1}=717.625
	\]
	\begin{equation*}
		\begin{split}
			\Delta h & =h_2-h_1=C_P(T_2-T_1)=403879.35\text{J}        \\
			\Delta e & =e_2-e_1=C_v(T_2-T_1)=288485.25\text{J}        \\
			\Delta s & =s_2-s_1=C_v \ln \frac{T_2}{T_1}+C_v(\gamma-1)
			\ln \frac{P_2T_1}{P_1T_2}=995.41
		\end{split}
	\end{equation*}
\end{example}

\begin{example}
	考虑一个等熵流经过机翼,来流的状态是$T_\infty=245$K,
	$P_\infty=4.35\times 10^4 $pa,在机翼上一点的压强是
	$3.6\times 10^4 $pa.计算这一点的密度.

	由等熵关系
	\[
		\frac{P_2}{P_1}=\left(\frac{T_2}{T_1}\right)^{\gamma-1}
	\]
	得到机翼上该点的温度
	\[
		T=(\frac{P}{P_\infty})^{\frac{1}{\gamma-1}}T_\infty=114.93\text{K}
	\]
	于是
	\[
		\rho =\frac{P}{RT}=\frac{3.6\times 10^4}{287.05\times 114.93}=1.09
		\text{Kg/m}^3
	\]
\end{example}

\section{正激波}
在流动过程中,气流的主要参数有显著的,突跃的变化的那一个地方称为激波,是在同一个位置有
无穷多压缩马赫波叠加而成,当激波的波阵面与来流方向垂直时,称为正激波.
如右图,\marginpar{插图}波前参数为$P_1$,$\rho_1$,$V_1$,波后参数为
$P_2$,$\rho_2$,$V_2$,由连续方程有
\[
	\rho_1V_1=\rho_2V_2
\]
因为激波很薄,截面积基本上没有发生变化,由动量方程有
\[
	\rho_2V_2^2-\rho_1V_1^2=P_1-P_2
\]
即
\[
	\rho_2V_2^2+P_2=\rho_1V_1^2+P_1
\]
又波前波后总能量相等,于是有
\[
	\frac{1}{2 }V_1^2+\frac{a_1^2}{\gamma-1 }=\frac{1}{2 }V_2^2+\frac{a_2^2}{\gamma-1 }=
	\frac{\gamma+1 }{2(\gamma-1)}a_*^2
\]
在
\[
	\rho_2V_2^2+P_2=\rho_1V_1^2+P_1
\]
左边除以$\rho_2V_2$,右边除以$\rho_1V_1$得到
\[
	V_2+\frac{P_2}{\rho_2V_2}=V_1+\frac{P_1}{\rho_1V_1}
\]
于是
\begin{equation}
	\begin{split}
		V_1-V_2 & =\frac{P_2}{\rho_2V_2}-\frac{P_1}{\rho_1V_1}                                                     \\
		        & =\frac{a_2^2}{\gamma V_2}-\frac{a_2^2}{\gamma V_1 },\left(\frac{P}{\rho}=RT,a^2=\gamma RT\right)
	\end{split}
	\label{eq:14}
\end{equation}
又
\[
	\frac{1}{2 }V^2+\frac{a^2}{\gamma-1 }=\frac{a_0^2}{\gamma-1 }=\frac{1}{\gamma-1 }
	\frac{\gamma+1 }{2 }a_*^2=\frac{\gamma+1 }{2(\gamma-1 )}a_*^2,(a_*^2=\frac{2}{\gamma+1 }a_0^2)
\]
有
\begin{equation}
	\begin{split}
		a_1^2 & =\frac{\gamma+1 }{2 }a_*^2-\frac{\gamma-1 }{2 }V_1^2 \\
		a_2^2 & =\frac{\gamma+1 }{2 }a_*^2-\frac{\gamma-1 }{2 }V_2^2
	\end{split}
	\label{eq:15 }
\end{equation}
代入到
\[
	V_1-V_2=\frac{a_2^2}{\gamma V_2}-\frac{a_1^2}{\gamma V_1 }
\]
有
\begin{equation}
	\begin{split}
		V_1-V_2 & =\frac{\gamma+1 }{2\gamma V_2}a_*^2
		-\frac{\gamma-1 }{2\gamma}V_2-\frac{\gamma+1 }{2\gamma V_1}a_*^2+
		\frac{\gamma-1 }{2\gamma}V_1                            \\
		        & =\frac{\gamma+1 }{2\gamma}a_*^2\left(
		\frac{1}{V_2}-\frac{1}{V_1}\right)+
		\frac{\gamma-1 }{2\gamma}(V_1-V_2)                      \\
		        & =\left(\frac{\gamma+1 }{2\gamma}\frac{a_*^2 }
		{V_1V_2}+\frac{\gamma-1 }{2\gamma}\right)(V_1-V_2)
	\end{split}
	\label{eq:16}
\end{equation}
又$V_1\neq V_2$,只能是
\[
	\frac{\gamma+1 }{2\gamma}\frac{a_*^2 }{V_1V_2}+\frac{\gamma-1 }{2\gamma}=1
\]
即
\begin{empheq}[box=\bluebox]{equation}
	\frac{a_*^2}{V_1V_2}=1 \text{或} a_*^2=V_1V_2
\end{empheq}
用速度系数表示
\begin{empheq}[box=\bluebox]{equation*}
	\lambda_1\lambda_2=1
\end{empheq}
上式就是普朗特激波公式,即正激波前后的速度乘积为定值,即临界速度的平方.
\begin{notice}
	激波并没有加热或者降温气体,因此气体流过激波是一个绝热过程,但是
	不是等熵过程.
\end{notice}


从上式可以看出:超声速气流$(\lambda_1 >1)$经过正激波后变为亚声速气流
$(\lambda_2 <1)$,而且速度系数$\lambda_1$越大则速度系数$\lambda_2$就越小.
应当指出,从亚声速流经过正激波变成超声速流是不可能的.

将式
\[
	\lambda^2=\frac{(\gamma+1)M^2}{2+(\gamma-1)M^2}
\]
代入到
\[
	\lambda_1\lambda_2=1
\]
即可得到
\begin{empheq}[box=\widefbox]{equation* }
	M_2^2=\frac{1+\frac{\gamma-1 }{2 }M_1^2 }
	{\gamma M_1^2-\frac{\gamma-1 }{2 }}
\end{empheq}
当$M_1=1 $时,$M_2=1 $;当$M_1\rightarrow \infty$,
$M_2\rightarrow\sqrt{\frac{\gamma-1 }{2\gamma}}$.

由连续方程$\rho_1V_1=\rho_2V_2 $和
\[
	\lambda^2=\frac{(\gamma+1)M^2}{2+(\gamma-1)M^2}
\]
可以得到正激波前后的密度比
\begin{empheq}[box=\widefbox]{equation* }
	\frac{\rho_2}{\rho_1}=\frac{V_1}{V_2}=
	\frac{\lambda_1}{\lambda_2}=\lambda_1^2
	=\frac{(\gamma+1)M_1^2}{(\gamma-1)M_1^2+2 }
\end{empheq}
再由过程的绝热性,$T_{01}=T_{02}$,以及静温和总温之间的关系
可以得到正激波前后的温度比为
\begin{empheq}[box=\widefbox]{equation*}
	\frac{T_2}{T_1}=\frac{T_{01}}{T_1 }\cdot \frac{T_2}{T_{02}}
	=\frac{2+(\gamma-1)M_1^2}{(\gamma+1)M_1^2}
	\left(\frac{2\gamma}{\gamma-1 }M_1^2-\frac{\gamma-1 }{\gamma+1 }\right)
\end{empheq}
于是正激波前后的压强比为
\begin{empheq}[box=\widefbox]{equation*}
	\frac{P_2}{P_1}=\frac{\rho_2 T_2}{\rho_1 T_1}=
	\frac{2\gamma}{\gamma+1 }-\frac{\gamma-1 }{\gamma+1}
\end{empheq}
也可以写成
\begin{empheq}[box=\widefbox]{equation* }
	\frac{\Delta P }{P_1}=\frac{P_2-P_1}{P_1 }\frac{2\gamma }{\gamma+1 }(M_1^2-1)
\end{empheq}

因为流动时超声速,$M_1>1 $,所以$\Delta P>0 $

超声速气流经过正激波的变化是绝热不等熵的.客观上,
激波层内速度梯度非常大,而实际流体总有粘性,因此必有摩擦发生
,所以经过激波熵必有所增加.

正激波前后总压变化为
\begin{empheq}[box=\widefbox]{equation*}
	\sigma=\frac{P_{02}}{P_{02}}=
	\left[1+\frac{2\gamma}{\gamma+1 }(M_1^2-1)\right]^{-\frac{1}{\gamma-1 }}
	-\left[\frac{(\gamma+1)M_1^2}{(\gamma-1)M_1^2+2 }\right]^{\frac{\gamma}{\gamma-1 }}
\end{empheq}

因为过程绝热,总温相等,所以总密度之比和总压之比相等. 上述
正激波前后关系式说明,正激波前后流动的参数的比值都只取决于
波前的马赫数$M_1$和比热比$\gamma$. 而马赫数$M_1$值越大,
激波突跃变化就越强,熵增也就越大.

\begin{notice}
	不论是正激波还是斜激波,激波前都是超声速流.
	斜激波后一般还是超声速,但是马赫数要下降.
	正激波后是亚声速流.
\end{notice}

下面总结讨论的气体如无特别指出都是量热完全气体.
\begin{summary}
	\begin{enumerate}
		\item 满足$P=\rho R T $的气体叫做{\color{red}理想气体或者热完全气体}.
		\item 定压热容$C_P$和定容热容$C_v$都是常数的气体叫做{\color{red}量热
		      完全气体}.
		\item {\color{red}绝热}是一个过程,即在这个过程中不从外界吸热也不向
		      外界放热,绝热过程总温不变,总焓和总内能也不变.
		\item {\color{red}等熵}是一个过程,即在这个过程中熵增为0,该过程全流场总压和
		      总密度都是常数.
		\item {\color{red}可逆过程}是指气体从状态1变化到状态2,然后在从状态2变回
		      状态1的过程对外界没有造成任何影响.强调是否造成影响的过程是从状态
		      2变回状态1的过程.
		\item 气流流过激波的过程是一个绝热过程(时间很短,与外界没有热交换),
		      但不等熵.
		\item 流过激波后,压强,密度,温度,熵都要{\color{red}增加};马赫数,速度和总压
		      {\color{red}降低};总温和总焓
		      {\color{red}不变}.
		\item 量热完全气体的绝热过程,内能和焓都只是温度的函数,满足
		      \begin{empheq}[box=\bluebox]{align*}
			      e&=e(T)=C_v T \\
			      h&=h(T)=C_P T
		      \end{empheq}
		\item {\color{red}流动定常绝热},满足
		      \begin{empheq}[box=\bluebox]{equation*}
			      \frac{1}{2 }V^2+h=h_0
		      \end{empheq}
		\item {\color{red}声速}计算公式
		      \begin{empheq}[box=\bluebox]{equation*}
			      a=\gamma R T
		      \end{empheq}
		\item {\color{red}流动等熵}则满足
		      \begin{empheq}[box=\bluebox]{equation*}
			      \frac{P_2}{P_1}=\left(\frac{\rho_2}{\rho_1}\right)^\gamma
			      =\left(\frac{T_2}{T_1}\right)^{\gamma-1}
		      \end{empheq}
		\item {\color{red}熵增}的计算
		      \begin{empheq}[box=\bluebox]{equation*}
			      \Delta s=s_2-s_1=C_v \ln \frac{T_2}{T_1}+R \ln \frac{\rho_1}{\rho_1}
		      \end{empheq}
		      或
		      \begin{empheq}[box=\bluebox]{equation*}
			      \Delta s=s_2-s_1=C_P \ln \frac{T_2}{T_1} -R \ln \frac{P_2}{P_1}
		      \end{empheq}
	\end{enumerate}
\end{summary}





