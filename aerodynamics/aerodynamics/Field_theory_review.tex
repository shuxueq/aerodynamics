% ! TEX root = ../mechanics.tex

\chapter{场论基础知识}

在介绍场论基础知识之前,这里不加约定地
将速度场$\mathbf{V}$认为是矢量场,而密度场
$\rho$,压强场$P$,温度场$T$等是标量场.如无
特殊说明,均认为上述参数是定常的.在引入新的
场变量时,用加粗字母表示矢量,不加粗字母表示数量.
比如$\mathbf{V}$时矢量场,但是$V$时标量场.
\section{梯度,旋度,散度}
\subsection{梯度计算}
对于标量场 $\rho=\rho(x,y,z)$,它的
梯度是
$$
    \nabla \rho=\frac{\partial \rho}{\partial x}\mathbf{i}
    +\frac{\partial \rho}{\partial y}\mathbf{j}
    +\frac{\partial \rho}{\partial z}\mathbf{k}
$$
标量场的梯度是一个向量,是这个标量函数下降最快的方向.
记作$\grad \rho$.

\subsection{旋度计算}
对于矢量场 $\mathbf{V}=u \mathbf{i}+
v \mathbf{j} +w \mathbf{k}$,它的旋度是
\begin{equation*}
    \nabla\times \mathbf{V}=
    \begin{vmatrix}
        \mathbf{i} & \mathbf{j}                   & \mathbf{k} \\
        \frac{\partial}{\partial x}
                   & \frac{\partial }{\partial y}
                   & \frac{\partial }{\partial z}              \\
        u          & v                            & w
    \end{vmatrix}
\end{equation*}
记作$\rot\mathbf{V}$或者$\curl\mathbf{V}$.
\begin{note}
标量场梯度的旋度恒等于0.
\end{note}

  
\subsection{散度}
对于向量场$\mathbf{V}$,它的散度是
\[
  \nabla\cdot\mathbf{V}=
  \frac{\partial u}{\partial x}+
  \frac{\partial v }{\partial y}+
  \frac{\partial w }{\partial z}
\]
记作$\dive\mathbf{V}$.
\begin{notice}
梯度计算是对于标量场,而散度,旋度均是对于
向量场.梯度和旋度的计算结果是向量,而散度
的计算结果是标量.
\end{notice}

\section{斯托克斯公式,散度定理,梯度定理}
假设有一个封闭曲面$S$,$\mathbf{A}$是一个在
封闭曲面$S$上有定义的矢量场.

根据斯托克斯公式有(该公式中$S$是非封闭曲面,
$C$是非封闭曲面$S$的边界曲线)
\[
  \oint_C \mathbf{A}\mathrm{d}\mathbf{r}=
  \iint_S (\nabla\times \mathbf{A})\cdot \mathbf{n}
  \mathrm{d}S
\]
也就是矢量场$\mathbf{A}$沿非封闭曲面$S$的边界曲线$C$
的积分等于$\mathbf{A}$的旋度在非封闭曲面上的积分.

根据散度定理有
\[
  \oiint_S \mathbf{A}\cdot \mathbf{n}\mathrm{d}S=
  \oiiint_v(\nabla \cdot \mathbf{A})\mathrm{d}v 
\]
也就是矢量场$\mathbf{A}$在封闭曲面上面的积分等于
这个矢量场在由这个封闭曲面包裹的闭空间中对体积的积分.

对于一个标量场$P$,由梯度定理有
\[
  \oiint_S P \mathbf{n}\mathrm{d}S=
  \oiiint_v\nabla P \mathrm{d} v 
\]

\begin{example}
有一个矢量场$\mathbf{V}=(x^2+y^2)\mathbf{i}+
(x^2-y^2)\mathbf{j}$
求$\mathbf{V}$的旋度,散度,散度的梯度.

\begin{equation*}
  \begin{split}
    \curl\mathbf{V}&= \left(\frac{\partial}{\partial x} (x^2+y^2)-
    \frac{\partial }{\partial y}(x^2-y^2)\right)\mathbf{k} \\ 
                   &=(2x+2y)\mathbf{k}\\
    \dive\mathbf{V}&=\frac{\partial }{\partial x}(x^2+y^2)+
    \frac{\partial }{\partial y}(x^2-y^2) \\ 
    &=2x-2y\\
\grad (\dive \mathbf{V})&=\frac{\partial}{\partial x}(2x-2y)\mathbf{i}+\frac{\partial}{\partial y}(2x-2y)
\mathbf{j}\\
&=2\mathbf{i}-2\mathbf{j}
\end{split}
\end{equation*}
\end{example}

\begin{notice}
  标量场和矢量场的差别在于,标量场是没有方向的,
  是一个数量场,也就是$\rho=\rho(x,y,z)$,是关于
  $x $,$y $,$z$的函数.矢量场是由向量组成的,是
  有方向的,也就是$\mathbf{V}=u(x,y,z)\mathbf{i}+
  v(x,y,z)\mathbf{j}+w(x,y,z)\mathbf{k}$,
  每个分量都是位置的函数.
\end{notice}

\section{并矢}
将两个矢量写在一起称为并矢,如$\mathbf{VV}$.
给定矢量$\mathbf{\alpha}=(\alpha_1,\alpha_2,
\alpha_3)$,$\mathbf{\beta}=(\beta_1,\beta_2,
\beta_3)$.并矢的计算的定义是
\[
  \mathbf{\alpha}\mathbf{\beta}=
  \begin{bmatrix}
      \alpha_1	\\
       \alpha_2	\\
      \alpha_3	\\
  \end{bmatrix}
  \begin{bmatrix}
      \beta_1	& \beta_2	& \beta_3	\\
  \end{bmatrix}
  =\begin{bmatrix}
       \alpha_1\beta_1	& \alpha_1\beta_2	& \alpha_1\beta_3\\
       \alpha_2\beta_1	& \alpha_2\beta_2	& \alpha_2\beta_3	\\
       \alpha_3\beta_1	& \alpha_3\beta_2	& \alpha_3\beta_3	\\
   \end{bmatrix}
\]
对于速度场的并矢$\mathbf{V}=(u,v,w)$,其中$u$,$v$,$w$
都是关于$x$,$y$,$z$的函数,如果非定常,还是关于时间
$t$的函数.
\[
  \mathbf{VV}=\begin{bmatrix}
                  u^2	& uv	& uw	\\
                  vu	& v^2	& vw	\\
                  wu	& wv	& w^2	\\
              \end{bmatrix}
\]
速度场并矢的散度计算如下,
\begin{equation*}
  \begin{split}
    \nabla \cdot (\mathbf{VV})&=
  \begin{bmatrix}
      \frac{\partial }{\partial x}	& 
      \frac{\partial }{\partial y}  &
      \frac{\partial }{\partial z}    \\
  \end{bmatrix}
  \begin{bmatrix}
      u^2	& uv	& uw	\\
      vu	& v^2	& vw	\\
      wu	& vw	& w^2	\\
  \end{bmatrix}\\ 
                              &=
   \begin{bmatrix}
      \frac{\partial }{\partial x }(u^2+uv+uw)&
      \frac{\partial}{\partial y }(uv+v^2+vw)&
      \frac{\partial}{\partial z }(uw+vw+w^2)\\
   \end{bmatrix}
\end{split}
\end{equation*}
并矢的散度是一个向量.
\begin{note}
  并矢已经不是向量场,而是一个张量,是一个二阶张量
  函数.
\end{note}

\section{物质导数}
流体的状态参数是位置和时间的函数,对于速度
$\mathbf{V}=\mathbf{V}(x,y,z,t)$,加速度就是
速度$\mathbf{V}$对于时间的导数,但是流体的
位置也是关于时间的函数,因此引入物质导数
$\frac{\mathrm{D}}{\mathrm{D}t }$.

定义物质导数的计算
\[
  \frac{\mathrm{D}}{\mathrm{D}t }=u \frac{\partial }
  {\partial x}+v \frac{\partial }{\partial y }+
  w \frac{\partial }{\partial z}+
  \frac{\partial }{\partial t }
\]
用梯度算子可以写成
\[
  \frac{\mathrm{D}}{\mathrm{D}t }=
  \frac{\partial }{\partial t }+
  \mathbf{V} \cdot \nabla 
\]
物质导数的物理意思是运动流体质点的某个量
随时间的变化率.$\frac{\partial }{\partial t }$是
当地导数,物理含义是确定空间点上的某个量随时间的
变化率.$\mathbf{V}\cdot\nabla$是牵连导数,物理
含义是具有空间不均匀流场中,由于质点的位置变化而
导致某个量随时间的变化.

速度的物质导数就是加速度.

\subsection{速度散度的物理意义}
速度散度也可以表示为
\[
  \nabla \cdot \mathbf{V}=\frac{1}{\delta v }
  \frac{\mathrm{D}(\delta v )}{\mathrm{D}t }
\]
其物理意义是,标定流体微团在运动过程中体积对时间
的变化率就是速度的散度.



