% ! TEX root = ../mechanics.tex

\chapter{不可压无粘流}
无粘流的动量方程就是欧拉方程.

对于不可压定常,无旋流的伯努利方程是
\[
	\frac{1}{2}\rho V^2+P +\rho U=
	\text{常数}
\]
上式左边分别是单位质量流体所具有的动能,压力
能,和位能(重力势),这三种能量统称为机械能.
它们三者之间可以相互转换,但是总和是保持不变的.

\begin{notice}
	在空气流动问题中,重力(重力势)可以忽略,于是伯努利方程
	可以写成
	\[
		P+\frac{1}{2}\rho V^2=\text{常数}
	\]
	其中,$\frac{1}{2}\rho V^2$称为动压,
	$P$称为静压,等号右边的常数称为总压($P_0$).
	总压是无粘流速度为零的点(驻点)的压强.
\end{notice}

对于有旋流动,伯努利方程的条件是沿流线成立,即
同一条流线总压相等.

\section{理想不可压无旋流动的控制方程}
不可压位流的控制方程是
\[
	\frac{\partial ^2 \Phi}{\partial x^2}+
	\frac{\partial ^2 \Phi}{\partial y^2}+
	\frac{\partial ^2 \Phi}{\partial z^2}=
  0 
\]
称为拉普拉斯方程.

{\bfseries 边界条件\index{边界条件}}就是流场的边界对流动
规定的条件.边界条件有三种,分别是
\begin{enumerate}
  \item 第一类边值问题,又称狄利克雷问题,
    即在边界上给定$\Phi$的值.
  \item 第二类边值问题,又称纽曼问题,
    即在边界上给定$\frac{\partial \Phi}{\partial n }$
    的值.
  \item 第三类边值问题,即混合边值问题,又称庞加莱问题,
    即在一部分边界上给定$\Phi$值,另一部分边界
    给定$\frac{\partial \Phi}{\partial n }$值. 
\end{enumerate}
对于理想不可压流的平面定常无旋流动中流函数满足的控制
方程是
\[
  \frac{\partial ^2 \Psi}{\partial x^2}+
  \frac{\partial ^2 \Psi }{\partial y^2}=
  0 
\]
\begin{note}
 拉普拉斯方程是一个线性方程,它的解满足叠加定理.
\end{note}

\section{拉普拉斯方程的基本解}
\subsection{直匀流}
{\bfseries 直匀流(uniform stream)\index{直匀流}}
是一种最简单的无旋流动,其中任何一点的流速都是一样的.
它的流函数是
\[
  \Phi=ax+by
\]
速度分量是
\begin{equation*}
  \begin{cases}
    u&=\frac{\partial \Phi}{\partial x}=a \\ 
    v&=\frac{\partial \Phi}{\partial y}=b
  \end{cases}
\end{equation*}
流函数是
\[
  \Psi=-bx+ay
\]

\subsection{点源}
{\bfseries 正源(source)\index{点源}}是从流场某点
有一定的流量流向四面八方的流动.{\bfseries
负源(sink)}则相反,也称为汇.

\begin{note}
 点源流动只有径向速度$V_r$,没有周向速度$V_\theta$.
\end{note}

记半径$r$处的流速为$V_r$,则源的总体积流量是$Q=2
\pi r V_r$,是一个常数.
所以
\[
  V_r=\frac{Q}{2\pi}\frac{1}{r }=\frac{Q}{2\pi}
  \frac{1}{\sqrt{x^2+y^2}}
\]
\begin{note}
 源的径向流速与$\theta$无关,$Q$称为源强.
\end{note}
点源的流函数是
\[
  \Psi=\frac{Q}{2\pi}\theta=\frac{Q}{2\pi}\arctan
  \left(\frac{y}{x}\right)
\]

点源的位函数是
\[
  \Phi=\frac{Q}{2\pi}\ln r =\frac{Q}{2\pi}
  \ln \sqrt{x^2+y^2} 
\]


