% ! TEX root =../mechanics.tex

\chapter{基本方程}
\section{连续方程}
\subsection{微分形式}
连续方程反应了质量守恒这个基本定律,
也就是流入控制体的质量和流出控制体
的质量(负质量)以及控制体内部的质量
和是保持不变的.

也可以描述为,随流体运动的流体微团
内的流体质量是保持不变的.假设流体
微团的体积为$\delta v$,质量为
$\rho \delta v$,流速是$\mathbf{V}$,
流体微团内部的质量是定常的,有
\[
	\frac{\mathrm{D}(\rho \delta v)}{\mathrm{D} t }=0
\]
因此有
\[
	\frac{\mathrm{D}\rho}{\mathrm{D}t }+\rho
	\nabla \cdot \mathbf{V}=0
\]
这就是连续方程.
可以写成
\[
	\frac{\partial \rho}{\partial t }+
	\nabla \cdot (\rho \mathbf{V})=0
\]
这就是流场中某点的流动变量之间的关系.

\subsection{积分形式}
单位时间内,流过微元面$\mathrm{d} S $的
流体质量就是流过该微元面的质量流量
$\dot{m}$,单位是Kg/s.
连续方程的积分形式用质量表述为
\[
	\dot{m}=\rho V_n \mathrm{d}S=
	\rho \mathbf{V}\cdot \mathbf{n}
	\mathrm{d} S
\]
\begin{note}
	对于不可压流可以用体积流量,原因
	请自行思考.
\end{note}

与质量流量相关的概念是质量通量密度,
定义为单位面积上的质量流量.用
$\dot{m}_A$表示.即
\[
	\dot{m}_A=\frac{\dot{m}}{\mathrm{d} S }=
	\rho V_n =\rho \mathbf{V} \cdot \mathbf{n}
\]
单位是$\mathrm{Kg /(s\cdot m^2)}$.

对于位置固定的控制体来说,质量守恒的描述
就是控制体内增加的流体质量+净流出控制体
的流体质量$=0$.就是
\[
	\oiiint_v \left[\frac{\partial \rho}{\partial t }
		+\nabla \cdot (\rho \mathbf{V})\right]\mathrm{d}v
	=0
\]
由于控制体任取,所以被积函数恒等于0,也就是前面的
微分形式.

\begin{notice}
	\begin{enumerate}
		\item 对于定常流动,所有流动参数对时间的偏导数都
		      等于0.
		      连续方程可以写为
		      \[
			      \nabla \cdot (\rho\mathbf{V})=0
		      \]
		      和
		      \[
			      \oiint_S \rho \mathbf{V}\cdot \mathbf{n}
			      \mathrm{d} S=0
		      \]
		\item 对于不可压流,密度是常数,连续方程可以写为
		      \[
			      \nabla \cdot \mathbf{V}=0
		      \]
		      和
		      \[
			      \oiint_S \mathbf{V} \cdot \mathbf{n}
			      \mathrm{d} S=0
		      \]
	\end{enumerate}
\end{notice}

\section{动量方程}
动量方程描述的是流体在流动过程中动量守恒的规律.

流体受到的外力等于单位时间内净流入控制体的动量和
控制体内部动量的增量.

\begin{notice}
	流体微团或控制体受到的外力有两个来源:
	\begin{enumerate}
		\item 重力,电磁力等,称为彻体力.
		\item 压力,粘性力,剪切力等,称为表面力.
	\end{enumerate}
\end{notice}

\subsection{微分形式}
动量方程的微分形式是
\[
	\frac{\partial (\rho \mathbf{V})}{\partial t }+
	\nabla \cdot (\rho \mathbf{VV})=
	-\nabla P+\rho \mathbf{f}+\nabla  \cdot
	\mathbf{\tau}
\]
上式左边是动量的变化率,右边是控制体受到的合力.
$\mathbf{f}$是控制受到的彻体力强度,如重力加速
度$g$.

\subsection{积分形式}
位置固定的控制体受到的彻体力大小是
\[
	\oiiint_v \rho \mathbf{f}\mathrm{d} v=\text{彻体力}
\]
压强力大小是
\[
	-\oiint_S P \mathbf{n}\mathrm{d} S=\text{压力}
\]
粘性力大小是
\[
	\oiint_S \mathbf{\tau}\cdot \mathbf{n}\mathrm{d}S=
	\text{粘性力}
\]
动量方程可以写成
\[
	\text{动量变化率}=G_1+G_2
\]
其中,$G_1$是单位时间内净流出控制面的流体质量所携带的
总动量;$G_2$是控制体内部因流场的非定常特性而产生的动
量的当地变化率.

那么
\[
	G_1=\oiint_S \mathbf{V}(\rho \mathbf{V}\cdot
	\mathbf{n}\mathrm{d}S)
\]
\[
	G_2=\frac{\partial }{\partial t }\oiiint
	_v \rho \mathbf{V}\mathrm{d}v
\]
于是就可得到动量方程的微分形式
\[
	\oiiint_v \frac{\partial (\rho \mathbf{V})}{
		\partial t}\mathrm{d} v +
	\oiint_S \mathbf{V} (\rho \mathbf{V}\cdot \mathrm{d}
	S)=\oiiint_v \rho \mathbf{f} \mathrm{d} v -
	\oiiint_v \nabla  P \mathrm{d}v +
	\oiiint_v \nabla \cdot \mathbf{\tau} \mathrm{d}v
\]

\begin{notice}
	对于不考虑彻体力的定常、无粘流动,微分形式的动量
	方程可以写成
	\[
		\nabla \cdot (\rho \mathbf{VV})=
		-\nabla  P
	\]
	积分形式可以写成
	\[
		\oiint_S \mathbf{V}(\rho \mathbf{V}\cdot
		\mathbf{n} \mathrm{d} S)=-\oiint_S P \mathbf{n}
		\mathrm{d} S
	\]
\end{notice}
\begin{note}
	压强力前面有负号的原因是,压强总是指向控制体内部
	,与面法线方向相反
	,故压强力总有负号.当然,具体问题具体分析,按照动量
	方程的本质列出相应的方程,这些矢量方程过于复杂.
\end{note}


\section{能量方程}
能量方程描述的是控制体能量守恒这个规律,即机械能和
内能守恒.
\begin{note}
	后面还会介绍焓这个概念.
\end{note}

\section{描述流体运动的方法}
\subsection{欧拉法和拉格朗日法}
描述流体运动的方法有两个,欧拉法和拉格朗日法.
欧拉法研究固定位置的流体区域的流动情况.
拉格朗日法则是追踪一个流体微团的运动情况.
\begin{note}
	由于流体运动微团太多,而实际只需要一定范围内的
	流动情况,因此描述流体运动常用欧拉法.
\end{note}

\subsection{流线(streamline)\quad 迹线
	(pathline)\quad 脉线(streakline)}
\begin{enumerate}
	\item 流线:是流体中的一条瞬时曲线,其上各点
	      的切线与该点的速度方向相同.
	\item 迹线:同一流体微团在不同时刻的位置所连
	      成的曲线.
	\item 脉线:在某一时间间隔内相继经过空间一固
	      定点的流体质点依次串连起来而成的曲线.
\end{enumerate}
\begin{note}
	流线、迹线、脉线只有当流动定常的时候才重合,一般
	情况下不重合.
\end{note}
\begin{enumerate}
	\item 流线方程\\
	      \[
		      \mathrm{d}\mathbf{r}\times \mathbf{V}=\mathbf{0}
	      \]
	      也就是
	      \begin{equation*}
		      \begin{vmatrix}
			      \mathbf{i}  & \mathbf{j}  & \mathbf{k}  \\
			      \mathrm{d}x & \mathrm{d}y & \mathrm{d}z \\
			      u           & v           & w           \\
		      \end{vmatrix}=\mathbf{0}
	      \end{equation*}
	      就是
	      \[
		      \frac{\mathrm{d}x}{u}=\frac{\mathrm{d}y}{v }=
		      \frac{\mathrm{d}z}{w}
	      \]
	\item 迹线方程 \\
	      由定义有
	      \begin{equation*}
		      \begin{cases}
			      \frac{\mathrm{d}x}{\mathrm{d}t } & =u \\
			      \frac{\mathrm{d}y}{\mathrm{d}t } & =v \\
			      \frac{\mathrm{d}z}{\mathrm{d}t } & =w
		      \end{cases}
	      \end{equation*}
	      整理可得
	      \[
		      \frac{\mathrm{d}x}{u}=\frac{\mathrm{d}y}{v }=
		      \frac{\mathrm{d}z}{w}=\mathrm{d} t
	      \]
\end{enumerate}

流管是一个和流线相关的概念.在流场任意选取一条不为流线
且不自交的封闭曲线,经过该封闭曲线的每一点作流线,所有
这些流线的集合,所构成的管状曲面称为流管(stream tube).
流管是由流线组成,
因此流线不能流出或流入流管表面.

\begin{note}
	同一流场中的流线不能相交,原因是同一点只有一个速度
	方向.流线相当于一堵墙,流体不能跨过流线,原因是流线
	的切线方向和流体的流动速度平行,不能相交.
\end{note}

\subsection{角速度\quad 角变形率}
流体微团在三维空间中的角速度是
\[
	\mathbf{\omega}=\frac{1}{2 }\left[
		\left( \frac{\partial w}{\partial y}-
		\frac{\partial v}{\partial z}\right)\mathbf{i}
		+\left( \frac{\partial u}{\partial z}-
		\frac{\partial w}{\partial x}\right) \mathbf{j}+
		\left( \frac{\partial v}{\partial x}-
		\frac{\partial u}{\partial y}\right) \mathbf{k}
		\right]
\]

实际上流体微团的角速度恰好等于速度旋度的一半.
定义
\[
	\mathbf{\Gamma} =2\mathbf{\omega}
\]
也就是
\[
	\mathbf{\Gamma} =\curl \mathbf{V}=\nabla
	\times \mathbf{V}
\]

如果流场的旋度等于0,称为无旋流动;反之则为有旋
流动.
\begin{note}
	无旋流动只做纯粹的平移运动和变形运动,有旋运动还
	做旋转运动.
\end{note}
若对于二维的无旋流则满足
\[
	\mathbf{\omega_z}= \frac{\partial v}{\partial x}-
	\frac{\partial u}{\partial y}=0
\]

角变形随时间的变化率称为角变形率.
\begin{equation*}
	\begin{cases}
		\gamma_z & = \frac{\partial v}{\partial x}+
		\frac{\partial u}{\partial y}                \\
		\gamma_y & = \frac{\partial u}{\partial z}+
		\frac{\partial w}{\partial x}                \\
		\gamma_x & = \frac{\partial v }{\partial z}+
		\frac{\partial w}{\partial y}
	\end{cases}
\end{equation*}
\begin{note}
	角变形率的记忆方法,和该维度无关的两个速度分量对
	该速度分量无关的位置变量的偏导数的和.比如$\gamma_x$,
	和该维度无关的两个速度分量是$v$和$w$,与速度分量$v$
	无关的位置变量是$z$.
\end{note}
\begin{example}
	给定速度场,$u=\frac{y }{x^2+y^2}$,
	$v=\frac{-x }{x^2+y^2}$,计算其旋度和角变形率,
	角速度.

  \begin{equation*}
    \begin{split}
      \mathbf{\omega}&=\frac{1}{2 }\curl \mathbf{V}
      =\frac{1}{2}\nabla \times \mathbf{V}=
		\frac{1}{2}
		\begin{vmatrix}
			\mathbf{i}                  & \mathbf{j}                  & \mathbf{k} \\
			\frac{\partial}{\partial x} &
      \frac{\partial}{\partial y} &
		  \frac{\partial}{\partial z}              \\
			\frac{y }{x^2+y^2}          & \frac{-x }{x^2+y^2}         & 0          \\
		\end{vmatrix}\\ 
                     &=-
    \left[\frac{x^2+y^2-2x^2}{(x^2+y^2)^2}+
      \frac{x^2+y^2-2y^2}{(x^2+y^2)^2}\right]
      \mathbf{k}\\
                     &=\mathbf{0}
  \end{split}
\end{equation*}

显然,旋度$\mathbf{\Gamma}=\mathbf{0}$.

角变形率
\begin{equation*}
  \begin{split}
    \gamma_x&= 0+0=0\\ 
    \gamma_y&= 0+0=0 \\ 
    \gamma_z&= \frac{\partial}{\partial x}
    \frac{-x }{x^2+y^2}+
    \frac{\partial}{\partial y}
    \frac{y }{x^2+y^2}
    =\frac{-x^2-y^2+2x^2}{(x^2+y^2)^2}+
    \frac{x^2+y^2-2y^2}{(x^2+y^2)^2}
    =\frac{2x^2-2y^2}{(x^2+y^2)^2}
  \end{split}
\end{equation*}

当然,对于$x^2+y^2\neq 0$的流场都是无旋的.
\end{example}

\section{流函数和势函数(速度位)}

%\section{流函数\quad 势函数}
\subsection{流函数}
对于二维不可压流动,连续方程是$\nabla \cdot \mathbf{V}$
也就是
\[
  \frac{\partial u}{\partial x}+
  \frac{\partial v }{\partial y}=0
\]
如果存在$p(x,y)$和$q(x,y)$满足,
$\frac{\partial p }{\partial y}=
\frac{\partial q }{\partial x }$,则存在
一个函数使得
\[
  \mathrm{d}f=p \mathrm{d}x+q \mathrm{d}y 
\]
因此,存在一个函数$\Psi (x,y,t)$的全微分是
\[
  \mathrm{d}\Psi=u \mathrm{d}y-v \mathrm{d}x 
\]
而
\[
  \mathrm{d}\Psi =\frac{\partial \Psi}{\partial x }
  \mathrm{d}x+
  \frac{\partial \Psi }{\partial y }\mathrm{d}y 
\]
故有
\begin{equation*}
  \begin{cases}
    u&=\frac{\partial \Psi }{\partial y}\\ 
    v&=-\frac{\partial \Psi}{\partial x }
  \end{cases}
\end{equation*}
函数$\Psi(x,y,t)$称为流函数.
\begin{note}
对于二维定常不可压流动,若通过两给定点作流线,由此
两条流线所界定的流管的体积流量就是这两条流线上的
流函数数值之差.
\end{note}
\begin{example}
  给定二维不可压流动的速度分布$u=x^2-y^2$,
  $v=-2xy$,求流函数.

  \begin{equation*}
    \begin{split}
      \frac{\partial \Psi}{\partial y }&=x^2-y^2 \\ 
      \frac{\partial \Psi}{\partial x }&=-(-2xy)=2xy 
    \end{split}
  \end{equation*}
  \[
    \Psi(x,y)=\int \frac{\partial \Psi}{\partial x}
    \mathrm{d}x =\int 2xy \mathrm{d}x =
    x^2y+C(y) 
  \]
  式中$C$是和$x $无关的常数.
  而
  \[
    \frac{\partial \Psi }{\partial y }=
    \frac{\partial (x^2y+C(y)}{\partial y}=
    x^2-y^2
  \]
  两相对比,有$C'(y)=-y^2$,于是$C(y)=-\frac{y^3}{3 }+C $
  所以
  \[
    \Psi(x,y)=x^2y-\frac{y^3}{3}+C 
  \]
  一般常数$C $也可以不写.
\end{example}
\begin{note}
流函数的定义是在不可压流中定义的,而且是二维平面,不满足
条件则没有流函数.有旋流和无旋流都有流函数.
\end{note}
\subsection{势函数(速度位)}
无旋流动满足
\[
  \mathbf{\Gamma}=\nabla \times \mathbf{V}=\mathbf{0}
\]
而标量函数梯度的旋度恒等于0,于是存在一个标量函数
使得
\[
  \mathbf{V}=\nabla \cdot \Phi
\]
成立.
称$\Phi(x,y,z,t)$为速度势函数.
根据定义有
\begin{equation*}
  \begin{cases}
    u&=\frac{\partial \Phi}{\partial x}\\ 
    v&=\frac{\partial \Phi}{\partial y}\\ 
    w&=\frac{\partial \Phi}{\partial z}
  \end{cases}
\end{equation*}
\begin{note}
只有流动无旋才可以定义势函数,否则没有势函数.
\end{note}
\begin{example}
  已知二维流场分布,$u=x $,$v=-y $,求该流场的
  势函数.

  首先先计算该流场的旋度.
  \[
    \curl \mathbf{V}=
    \left[\frac{\partial (-y)}{\partial x }-
      \frac{\partial x }{\partial y }\right]
      \mathbf{k}=0
  \]
  流动无旋,因此存在势函数.

  于是有
  \[
    \mathrm{d}\Phi=x \mathrm{d}x -y \mathrm{d}y 
    =\mathrm{d}(\frac{x^2}{2 })-\mathrm{d}
    (\frac{y^2}{2 })
    =\mathrm{d}(\frac{x^2-y^2}{2 }+C)
  \]
  因此,$\Phi(x,y,t)=\frac{x^2-y^2}{2 }+C $,不同于
  流函数,这里常数$C$需要根据初值条件来求取.
\end{example}
\begin{note}
这里求势函数的方法是凑微分,和前面求流函数的方法不
同,两种方法都可以求解.
\end{note}

\begin{notice}
  对于二维不可压流动,流线就是流函数$\Psi$值相等
  的线.等位线($\Phi=$常数)和流线($\Psi=$常数)始终
  正交(驻点除外).
\end{notice}

\section{旋涡运动}
{\bfseries 涡线}是旋涡场中的一条瞬时曲线,其上各点的切线
与该点处的流体微团的旋转角速度方向相同.
涡线的微分形式方程是
\[
  \frac{\mathrm{d}x }{\mathbf{\omega}_x}=
  \frac{\mathrm{d}y }{\mathbf{\omega}_y}=
  \frac{\mathrm{d}z }{\mathbf{\omega}_z}
\]

取一条封闭曲线,速度线积分的值定义为{\bfseries
速度环量},
\[
  \Gamma=\oint_C \mathbf{V}\cdot \mathrm{d}\mathbf{r}
\]
速度环量取逆时针方向为正向.
\begin{note}
有些也取顺时针为正向.
\end{note}

把流场中由于旋涡存在而产生的速度称为{\bfseries
诱导速度},即
\[
  \mathrm{d}\mathbf{V}_P=
  \frac{\Gamma}{4\pi}
  \frac{\mathrm{d}\mathbf{r}_1\times \mathbf{r}_{1P}}
  {r^3_{1P}}
\]
上式也就是毕奥-萨伐尔定律.

\subsection{亥姆霍兹定理}
关于旋涡运动,有亥姆霍兹的三个定理.
\begin{enumerate}
  \item 在同一瞬间,沿涡线或涡管的强度不变.
  \item 涡管不能在流体中中断;只能在流体边
    界上中断或形成合圈.
  \item 如果流体是理想的,正压的且彻体力有
    势,那么涡的强度不随时间变化,既不会增强,
    也不会削弱. 
\end{enumerate}

