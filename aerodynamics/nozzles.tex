% ! TEX root = ../aerodynamics.tex
\chapter{喷管中的可压流动}
\section{准一维流动的控制方程}
喷管中的流动依然满足连续方程,动量方程
和能量方程。

首先是满足连续方程,即
$$
\rho u A=\operatorname{const}
$$
微分形式是
$$
\mathrm{d}\left(\rho u A\right)=0
$$
整理得到
$$
\frac{\mathrm{d}\rho}{\rho}
+\frac{\mathrm{d}u}{u}
+\frac{\mathrm{d}A}{u}=0
$$
因为
\[
  \mathrm{d}P=-\rho u \mathrm{d}u 
\]
上式即欧拉方程。
再由能量方程
\[
  h+\frac{u^2}{2}=const
\]
微分形式是
\[
  \mathrm{d}h+u \mathrm{d}u =0 
\]
综上,上述方程就是一维喷管中的控制
方程。

声速的定义
\[
  a^2=\frac{\mathrm{d}P }{\mathrm{d}\rho}
\]
联立连续方程的微分形式,以及
\[
  \frac{\mathrm{d}P }{\rho}= 
  \frac{\mathrm{d}P }{\mathrm{d}\rho}
  \frac{\mathrm{d}\rho}{\rho}= 
  - u \mathrm{d}u 
\]
可以得到
\[
  \frac{\mathrm{d}\rho}{\rho}
  = -\frac{u \mathrm{d}u }{a^2}
  = -M^2 \frac{\mathrm{d}u }{u}
\]
这个方程很重要,我们可以从中
得到一下几点重要信息:
\begin{enumerate}
  \item 对于亚声速流动,在扩
    张管道中,流速是减小的;
    在收缩管道中,流速是增加
    的。
  \item 对于超声速流动,在扩
    张管道中,流速是增加的;
    在收敛管道中,流速是减小
    的
  \item 对于声速流动,即
  $M=1$。那么$M^{2}-1=0$,
  也就是说速度有微小的变化,
  即$\mathrm{d}u$存在,也
  不会引起$\mathrm{d}A$的
  变化,即$\mathrm{d}A=0$。
\end{enumerate}
请记住,气流在喷管中加速,都是
在发生膨胀,也就是气体本身的焓
在逐渐转化为动能。

想像一下,我们想要将静止的气体
膨胀加速到超声速,喷管应该是一
个先收缩再扩张形式的,这就是拉
瓦尔喷管。一般来说,收缩喷管和
扩张喷管都叫做拉瓦尔喷管。值得
注意的是并不是所有的收缩--扩张
喷管都能将亚声速气体加速到超声
速。原因我们在后面讲解。要让亚
声速气流加速到超声速,首先要让
在收缩喷管中加速到声速,然后再
接一个扩张喷管,自然,加速到声
速的位置就是收缩喷管面积最小的
位置,也是扩张喷管面积最小的位
置,我们把这个位置叫做
{\bfseries 喉道(throat)},
当然,喉道处气流的流速就是声速。
这是对于一个正常工作的喷管来说
的,这是因为收缩--扩张喷管并不
一定能将亚声速气流加速到超声速。
那么反过来,如果一股超声速气流
流过收缩--扩张喷管,会先在收缩
管道中减速到声速,然后再在扩张
管道中减速至亚声速。同样,在喉
道处达到声速。这样的装置我们叫
做
{\bfseries 进气道}。
\begin{note}
  喷管中的流动是等熵流动
\end{note}

\section{喷管中的流动}
前面我们讨论了喷管的形状对流速
的影响,接下来讨论喷管中的流动
情况。

考虑一个收敛--扩张喷管,在喉道
处的面积记作$A^*$,马赫数和速
度分别记作$M^*$和$u^*$。因为
喉道处是声速,所以$M^*=1$,
$u^*=a^*$。
喷管其他截面处的面积和马赫数分别
记作
$A$和$M$。
根据连续性方程,我们可以得到
\[
\rho^* u^* A^*=\rho u A   
\]
因为
$u^*=a^*$所以上式可以写作
\[
\frac{A}{A^*}=
\frac{\rho^*}{\rho}
\frac{a^*}{u}
=\frac{\rho^*}{\rho_0}
\frac{\rho_0}{\rho} 
\frac{a^*}{u}
\]
再由前面的可压流动的能量方程
\[
\frac{\rho^*}{\rho_0}=
\left(\frac{2}{\gamma+1}\right)
^{\frac{1}{\gamma-1}}
\]
和
\[
\frac{\rho_0}{\rho}=
\left(1+\frac{\gamma-1}{2}M^{2}\right)^{\frac{1}{\gamma-1}}
\]
以及
\[
\frac{u}{a^*}=
\sqrt{
  \frac{\frac{\gamma-1}{2}M^{2}}{1+\frac{\gamma+1}{2}M^{2}}
}
\]
代入得到
\[
\left(\frac{A}{A^*}\right)^2=
\left(\frac{2}{\gamma+1}\right)^{2}
\left(1+\frac{\gamma-1}{2}M^{2}\right)^{2}
\frac{1+\frac{\gamma-1}{2}M^{2}}{1+\frac{\gamma+1}{2}M^{2}}
\]
简化后得到
\[
\left(\frac{A}{A^*}\right)^{2}=
\frac{1}{M^{2}}
\left[\frac{2}{\gamma+1}
\left(1+\frac{\gamma-1}{2}
M^{2}\right)\right]^{
  \frac{\gamma+1}{\gamma-1}
}
\]
上面这个方程同样很重要,这个方程叫
做面积--马赫数关系。这就是说任何位
置处的马赫数都是这个截面面积和喉道
面积比值的函数。任何位置处的面积都
要比喉道处的面积大,
$A<A^*$的情况对于等熵流动在理论上
式不可能发生的。因此,对于上式来说
$\frac{A}{A^*}>1$。对于给定
$\frac{A}{A^*}$来说,
$M$有两个解,一个对应超声速,一个 
对应亚声速。在给定情况下,$M$取那
个值,取决于喷管入口和出口的压强。

想象你已经获得了一根收敛--扩张喷管,
简单将它放在你面前的桌子上,接下来 
会发生什么呢?空气会自动从喷管中流
出来吗?答案当然是不会啦!我们必须
要施加一个力来产生一定的加速度。事
实上这就是动量方程的本质,入口处空
气的动量是0 ,出口处空气的动量也必
须是0 。这里我们考虑的是无粘流体,
推动空气流动的力就变成了压强,也就
是喷管出口和入口的压强差,这个压力
会持续地提供冲量。回到桌子上的这个
喷管,只有当出口和入口存在压强差,
空气才会流入喷管。并且,入口处的压
强必须大于出口处的压强,即
$P_e<P_0$
其中$P_e$表示出口处的压强,$P_0$表
示入口处的压强。更重要的是如果我们
希望产生等熵流动,压比
$\frac{P_e}{P_0}$
必须和我们前面的面积--马赫数关系计算
得到的压比一样,否则就达不到我们所希
望的速度。
\begin{notice}
每个喷管都有一个设计压比,如果是按照
设计的压比工作,就是正常工作,也就是
说符合面积--马赫数关系,否则喷管内部
将产生激波,使得出口处的速度比理想的
速度要小。
\end{notice}

接下来我们讨论实际压比不等于理论压比
的情况。如果$P_e=P_0$, 那么喷管中的空
气将不会发生流动;如果$P_e$只比$P_0$
小一点点,比如$P_e=0.999P_0$,那么喷
管中将会产生一个流速很小的亚声速流动,
喷管中将会吹出一股小风。马赫数将会在
收敛喷管中变大一些,在喉道处达到最大
值,但是并没有达到声速,然后在扩张喷
管处马赫数减小。相应地,压强在收敛管
道中逐渐减小,在喉道处达到最小值,然
后在逐渐地增大,直到在出口处达到$P_{e,1}$。

假设出口处的压强进一步减小,气流在喷
管中的流动就会更快,但是喉道处的马赫
数依然小于1 。记这个时候出口处的压强
是$P_{e,2} $。如果继续减小出口处的压
强,这个时候,喉道处的马赫数就等于1 
了,但是喉道下游的马赫数依然小于1 。
记这个时候出口处的压强是$P_{e,3}$。

通过上面的讨论我们知道,只有一种等熵
流动的解满足超声速的情况。与之相反的
是,对于满足亚声速流动的解却有很多。
只要任何满足$P_0\geq P_e\geq P_{e,3}$
的出口压强$P_e$,喷管中都是亚声速流动
。因此分析收敛--扩张喷管中的两个关键
因素是$\frac{A}{A^*}$和$\frac{P_e}{P_0}$。

考虑喷管中质量流量的变化。随着出口压
强减小,气流的流速逐渐变大,质量流量
也逐渐变大。但是随着$P_e $减小,速度
$u_t $在变大,密度$\rho_t$在减小。然
而,速度增大的比例比密度减小的比例更
大,所以总体上来说,质量流量还是变大
的。现在如果出口压强进一步减小,减小
到比$P_{e,3} $还要小,喉道处的流动情
况就发生了变化,质量流量不在发生变化
。这是因为喉道处的马赫数不能大于1 ,
因此,如果出口处的压强进一步减小,喉道
处的压强始终保持为1 。结果是喉道处的
质量流量也保持为常数。形象地说,喉道
的上游就像是被冻住了一样。一旦在喉道
处达到了声速,喉道处的扰动就不能传递
到上游。因此喷管的收敛段就不再和出口
处的压强“ 交流” ,也不知道出口处的压强
还在减小。当喉道处气流达到声速,无论
出口处的压强减小到多小,喷管中的质量
流量都不再变化的情况叫做
{\bfseries 壅塞流(choked flow)}。

当出口处的压强降低到比$P_{e,3} $还小,
会发生什么呢?收敛段的情况如前所述,
但是扩张段会发生很多变化。首先,下
游将会出现超声速区域。然而,由于出口
处的压强还是太大而不允许整个扩张喷管
都是等熵流动。因此当
$P_{e,3}\geq P_e \geq P_{e,6}$
时($P_{e,6}$是指喷管设计的出口压强,
也就是在给定的$P_0$下,喷管中都是等
熵流动时,出口处的压强),将会在喉道
的下游形成一道正激波,此时出口处的压强
就变成了
$P_{e,4}(P_{e,3}\geq P_{e,4}\geq P_{e,6})$
。我们可以观察到正激波出现在距离喉道
$d$的下游处。 在正激波和喉道中间的区域,
气流已经被加速到了超声速,这部分流动符合
等熵流动的解。在正激波后,就变成
亚声速了,这股亚声速流将会随着它
向出口流动等熵地减速。相应地,压强
随着正激波出现而不连续地增加,并且
将随着气流减速进一步增加。因此在激
波的左右两侧都是等熵流动,但是跨越
激波后,出现了熵增。随着$P_e $继续
减小,正激波逐渐向出口处移动,直到
减小到$P_{e,5} $时,正激波恰好移动
到了出口处。这种情况下,流过整个喷
管的气流都是等熵的,出来喷管的出口
处。

前面我们讨论了出口处的压强,但是还
没讨论喷管出口下游的流动情况。
想象一下,喷管直接向环境排气。在很
多情况下,下游出口处环境的压强叫做
{\bfseries 背压(back pressure)},
记作$P_B $。当喷管出口处是亚声速的
时候,出口压强必须等于背压,$P_e=P_B$
,这是因为稳定的亚声速流动不能出现
压强的不连续。也就是说,出口是亚音速
的时候,环境的背压直接作用在出口的
气流上。只要$P_e\geq P_{e,3}$ 就有
$P_B=P_e $。同样的,$P_B=P_{e,4} $
和$P_B=P_{e,5} $。

如果我们将背压减小到比$P_{e,5} $还要
小会怎么样?当$P_{e,6}<P_B<P_{e,5}$
,背压依然高于喷管设计时的出口压强。
所以从出口处喷出的气流为了流向环境中,
气流必须以某种方式被压缩,使得压强
与背压相匹配。这些压缩过程时通过出口
处的斜激波系来完成的。当背压降低到
$P_B=P_{e,6} $时,出口压强和背压不
存在不匹配的情况。从喷管喷出的气流
平滑地稳定地向周围排除,没有经过任
何的激波。最后,当背压进一步减小,
即$P_B<P_{e,6} $的时候,气流必须进
一步膨胀,以减小压强和背压相匹配。
这些膨胀过程发生在出口处的一些膨胀
波系中。当出现这种情况时,我们说
喷管是{\bfseries 过膨胀(over-expanded)}的。
这就是说气流在喷管中膨胀得太多,必须
通过一些斜激波取达到比$P_e$更大的
背压。相反地,如果从喷管中喷出后,
必须经过一系列的膨胀波来加速,使之
与背压相匹配,即$P_{e,6}>P_B $,这
种情况叫做{\bfseries 欠膨胀(under-expanded)}
的。这就是说,气流在喷管中膨胀得不够,
导致出口压强比环境背压还要大,必须
通过一系列膨胀波来降压。

记住只要$P_B\leq P_{e,5}$时,管内
都是纯粹得等熵流动,没有激波出现,
而且都是完全膨胀的。前面的讨论给了
我们研究喷管中流动的性质,但是并没
有给出如何设计喷管轮廓的方法。这需
要用到三维流动的设计方法。

\begin{notice}

\hspace{2em}
相信你看到这里已经有一些迷糊了,出口
压强、入口压强、背压、喉道处的压强以
及一些列的出口压强也许已经分不清了吧!
如果你还能分清,那说明你已经把喷管中
流动的本质搞清楚了。如果你还没有理解,
请你继续阅读下面的内容,也许对你理解
有所帮助。

\hspace{2em}
首先,你需要弄清楚上面这些压强的概念,
即入口压强,出口压强,背压分别是什么。

\hspace{2em}
因为喷管中气流流动必须需要出口和入口
存在一定的压强差,所以要求$P_0>P_e$。
这是前提条件,如果不满足这个条件,气
流在喷管中不再流动,后面的讨论就没有
意义。如果说入口处的压强只比出口处的
压强稍大一些,首先气流肯定会在压强的
作用下向出口流去,但是由于压强差太小
了,所以气流的加速效应不太明显,在喉
道处并没有加速到声速。这里不要认为只
要是亚声速气流流过收敛--扩张喷管就会
被加速到超声速,这种理解是错误的。
这是因为出口压强有可能会阻止气流向外
流动,也就是说气流在喉道处加速到声速,
喉道处的压强小于出口压强,这样气流就
在一个逆压梯度区流动,如此气流必定减
速,动能转化为压力能,或者说是焓。
另外,气流的能量不一定能够支持气流加
速到超声速,加速过程本质上是气流的焓
转化为气流流动的动能,如果焓太小,这
个转化过程可能就不能持续下去。我们继
续刚刚的讨论。气流流到喉道处没有加速
到声速,然后就要流经扩张喷管,所以气
流是减速的,也就是说气流在流经收敛喷
管加速后又经扩张喷管减速。这时,我们
把出口处的压强记为$P_{e,1} $。
{{\color{noteorange}\bfseries 
这里我们把气流在喷管中全部等熵膨胀后,
出口处的压强记为$P_{e,6} $ }}。
这就是说,$P_{e,1}>P_{e,6} $,这是因
为气流在膨胀后压强必然减小,而全部等
熵膨胀压强就会减到最小,而这里并没有
完全等熵膨胀。

\hspace{2em}
接下来,让我们在减小出口处的压强,这
样在喉道处的马赫数依然小于1 ,注意这
种情况和前面那种情况的差别,第一种情
下,我们说出口处的压强只比入口处的压
强小一些,也就是说出口压强和入口压强
十分接近,只是满足了气流会从入口流向
出口。而这里还要满足喉道处的马赫数小
于1 。我们记这种情况下,出口的压强是
$P_{e,2} $。也许你还不能理解,你可以
理解成$P_{e,1}=0.9999P_0$,而比这个大
,同时满足喉道处马赫数压强小于1 这两
个条件的出口压强就是$P_{e,2} $,也就
是说$P_{e,2}> P_{e,1}$。注意,这两个
压强并没有明显的区分,只不过第二种情
况下,整体的流动速度比第一种大。

\hspace{2em}
如果再减小压强,使得喉道处的马赫数刚
好达到了1 ,也就是说喉道处是声速流动。
这个时候,记出口处的压强为$P_{e,3} $。
显然,$P_{e,3}>P_{e,2} $。这种情况下,
出口处还不能达到超声速,也许你会有疑
惑,明明喉道处都达到声速了,在扩张管
道中应该是膨胀加速的才对。但是喉道处
的压强比出口处的压强低,压强使得气流
不能往外流动,因此气流必须减速以匹配
出口压强,从而向外流动。注意,喷管中
等熵膨胀的超声速流动解只有一个,而等
熵膨胀的亚声速流动解有很多个,只要满
足$P_0>P_e>P_{e,3} $。

\hspace{2em}
如果继续减小压强,小于$P_{e,3} $。这
时,喷管中的气流流量不会像前面几种情
况一样,减小出口压强,气流流量会增加,
而是保持不变了。这时,继续减小压强,
就会出现堵塞,或者说是喷管中的气流像
被冻住了一样,记这个时候出口处的压强
是$P_{e,4} $。这时,收敛喷管中的流动
情况和上面描述的情况一样。在扩张喷管
中,气流将会被加速到超声速。但是,由
于$P_{e,4}>P_{e,6} $,所以气流不可能
在喷管中完全膨胀,所以在扩张段中必定
要减速,所以会出现一道正激波,使得气
流减速到亚声速,从而气流在扩张管道中
继续减速,直到达到出口处的压强$P_{e,
4}$。

\hspace{2em}
如果继续减小压强,扩张管道中的正激波
将会向出口处移动,直到刚好移动到出口,
记这个时候出口处的压强是$P_{e,5} $。
显然,$P_{e,5}>P_{e,6} $,这是因为这
种情况依然需要一道正激波来使气流增压
达到出口处的压强。

\hspace{2em}
如果再减小压强,气流将在喷管中完全膨
胀,出口处的压强就是$P_{e,6} $,此时
再研究出口处的压强没有意义了,因为出
口处的压强将保持不变,管内也不会出现
正激波,如果出口处的压强比$P_{e,6} $
还低,气流也无法膨胀到很高的速度,使
得压强降低得比$P_{e,6} $还低。这个时
候引入背压$P_B$。如果背压比$P_{e,6} 
$ 还要低,那么气流流出喷管后还有经历
一次膨胀,在喷管出口处产生一系列得膨
胀波,这个时候$P_B<P_{e,6} $。如果背
压大于$P_{e,6} $,那么气流就要增压,
必然减速,因为没有外界的能量注入。而
减速就需要激波,这里在喷管外部产生的
是一系列斜激波来使得气流减速增压。如
果出口处是亚声速流,那么$P_e=P_B $,
这是因为,亚声速不能产生激波使得压强
发生突变,也就是说亚声速流中压强是连
续变化的。所以$P_B=P_{e,1} $,$P_B=P
_{e,2} $,$P_B=P_{e,3} $,$P_B=P_{e,
4} $,$P_B=P_{e,5} $都成立。所以前面
所说的斜激波系将会在$P_{e,5}<P_e<P_B$
的条件下产生。

\hspace{2em}
至此就把喷管的流动情况都介绍完了,希
望你能全部理解。你只需要记住,出口处
的压强要匹配,出口压强要和背压匹配,
气流出口的压强是必须要达到的,我们的
讨论建立在这样的前提条件下。
\end{notice}




