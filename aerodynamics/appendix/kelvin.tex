% ! TEX root = ../mechanics.tex

\chapter{开尔文环量定理证明}
\label{kelvin theroem}
考虑无粘不可压流动,由环量定义有
\[
	\Gamma=\oint _c \mathbf{V}\cdot \mathrm{d} \mathbf{s}
\]
由于力学中,函数的性质都很好,所以微分运算和积分运算可以
直接交换次序。即
\[
	\begin{split}
		\frac{\mathrm{D} \Gamma}{\mathrm{D}t } & =\frac{\mathrm{D} }{\mathrm{D} t }\oint _c \mathbf{V} \cdot \mathrm{d}\mathbf{s}  \\
		                                       & =\oint _c \frac{\mathrm{D}}{\mathrm{D} t}(\mathbf{V}\cdot \mathrm{d} \mathbf{s})  \\
		                                       & =\oint _c \frac{\mathrm{D} \mathbf{V} }{\mathrm{D} t}\cdot \mathrm{d}\mathbf{s} +
		\mathbf{V}\cdot \frac{\mathrm{D}(\mathrm{d}\mathbf{s})}{\mathrm{D} t}
	\end{split}
\]
因为
\[
	\frac{\mathrm{D}(\mathrm{d}\mathbf{s})}{\mathrm{D}t}=\mathrm{d}\mathbf{V}
\]
V是一个单值函数,所以
\[
	\begin{split}
		\oint _c \mathbf{V} \cdot \frac{\mathrm{D}(\mathrm{d}\mathbf{s})}{\mathrm{D}t} & =
		\oint _c \mathbf{V}\cdot \mathrm{d}\mathbf{V}                                       \\
		                                                                               & =
		\oint _c (\mathrm{d} \frac{V^2}{2})                                                 \\
		                                                                               & =0
	\end{split}
\]
所以
\[
	\frac{\mathrm{D} \Gamma}{\mathrm{D}t}=\oint _c \frac{\mathrm{D} \mathbf{V}}{\mathrm{D}t}\cdot \mathrm{d}\mathbf{s}
\]
由动量方程,$\frac{\mathrm{D}\mathbf{V}}{\mathrm{D}t}$是加速度,也即单位质量流体受到的合外力。
所以
\[
  \frac{\mathrm{D}\mathbf{V}}{\mathrm{D}t}=-\frac{1}{\rho}\nabla P
\]
而
\[
  \begin{split}
  \nabla P \cdot \mathrm{d}\mathbf{s}&=(\frac{\partial P }{\partial x},\frac{\partial P}{\partial y},
  \frac{\partial P}{\partial z})\cdot (\mathrm{d}x,\mathrm{d}y,\mathrm{d}z)\\ 
                                     &=\frac{\partial P }{\partial x}\mathrm{d}x+
                                     \frac{\partial P}{\partial y}\mathrm{d}y+
  \frac{\partial P}{\partial z}\mathrm{d}z\\
                                     &=\mathrm{d}P(全微分定义)
  \end{split}
\]
于是
\[
  \begin{split}
    \frac{\mathrm{D} \mathbf{V}}{\mathrm{D} t}\cdot \mathrm{d}\mathbf{s}&=
    -\frac{1}{\rho}\nabla P \cdot \mathrm{d}\mathbf{s}\\                                                                      &=
     -\frac{1}{\rho}\mathrm{d}P
\end{split}
\]
于是
\[
  \oint _c 
    \frac{\mathrm{D} \mathbf{V}}{\mathrm{D} t}\cdot \mathrm{d}\mathbf{s}=
    \oint _c 
     -\frac{1}{\rho}\mathrm{d}P
     =0
\]
其中$\rho=常数$或者$\rho=\rho(P)$,即为不可压流动,或者是密度只是压强的函数的流动。

