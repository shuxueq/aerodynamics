% ! TEX root = ../mechanics.tex 

\usepackage{amsmath}
\usepackage{mathtools}
\usepackage{unicode-math}
%\usepackage{mathspec}
\usepackage[heading=true,scheme=chinese,linespread=1.5,zihao=-4,fontset=none]{ctex}
\usepackage[hidelinks]{hyperref}
\usepackage{tocloft}%目录
\usepackage[left=2.5cm,right=2cm,bottom=2cm,top=2cm]{geometry}%设置页面
\usepackage{fancyhdr}%设置版面
\usepackage{xcolor}
\usepackage{framed}
\usepackage{pdfsync}
\usepackage[overload]{empheq}
\usepackage{tikz}
\usepackage{tcolorbox}
\tcbuselibrary{breakable, skins, theorems}
\usepackage{witharrows}
\usepackage{manfnt}
\usepackage{enumitem}
\usepackage{makeidx}
\usepackage{appendix}
\usepackage{graphicx}
\usepackage{float}
\usepackage{xeCJK}
\usepackage{extarrows}
%\usepackage{autobreak}
\usepackage{fontawesome}
\usepackage{booktabs}%三线表
%\usepackage{svg}

%行内数学公式展示模式
\everymath{\displaystyle}
%行间公式允许跨页
\allowdisplaybreaks[4]
%数学算子定义
\DeclareMathOperator*{\dive}{div}
\DeclareMathOperator*{\curl}{curl}
\DeclareMathOperator*{\rot}{rot}
\DeclareMathOperator*{\grad}{grad}
\renewcommand{\grad}{\mathbf{grad}\,}
\renewcommand{\rot}{\mathbf{rot}\,}
\renewcommand{\curl}{\mathbf{curl}\,}
%颜色定义
\definecolor{titleblue}{RGB}{71, 143, 204}
\definecolor{notered}{RGB}{255, 25, 25}
\definecolor{noteorange}{RGB}{255, 134, 24}
\definecolor{main}{RGB}{0, 166, 82}
\definecolor{blue}{RGB}{0, 174, 247}

%页面版式设置
\fancyhf{}
\fancyfoot[c]{\color{titleblue}\small\thepage}
%\fancyfoot[C]{\textcolor{titleblue}{\thepage}}
\fancyhead[r]{\color{titleblue}\rightmark}
%\fancyfoot[RO,LE]{$\heartsuit$}
\renewcommand{\headrule}{\color{titleblue}
	\hrule width\headwidth height\headrulewidth
	\vskip-\headrulewidth}

%列表环境设置
\setlist[enumerate]{label=\color{titleblue}{\arabic* .}}
\setlist[enumerate,2]{label=\color{titleblue}{\alph*)}}

%制作索引
\makeindex

%字体设置
\setCJKmainfont[Mapping = fullwidth-stop,BoldFont=SourceHanserifSC-Bold.otf]{SourceHanserifSC-Regular.otf}
\setmathfont{XITS Math}
\setmathrm{XITS Math}
\setmainfont{Times New Roman}



%ctex宏包设置
\ctexset{
	%fontset=winfonts,
  punct=kaiming,
	tocdepth=1,%目录深度设为3级
	part/format+=\color{titleblue},
	chapter/format=\color{titleblue}\LARGE\centering\bfseries,
	section/format=\color{titleblue}\large\bfseries,
	subsection/format=\color{titleblue}\bfseries,
}

%
\xeCJKsetup{
  CJKmath=true,
  CJKspace=false,
}
\tcbset{
	common/.style={
			%fontupper=\citshape,
			lower separated=false,
			coltitle=white,
			colback=gray!5,
			boxrule=0.5pt,
			fonttitle=\bfseries,
			enhanced,
			breakable,
			top=8pt,
			before skip=8pt,
			attach boxed title to top left={
					yshift=-0.11in,
					xshift=0.15in},
			boxed title style={
					boxrule=0pt,
					colframe=white,
					arc=0pt,
					outer arc=0pt},
			separator sign={.},
			description delimiters = {(}{)},
		},
	exastyle/.style={
			common,
			colframe=main,
			colback=main!5,
			colbacktitle=main,
			overlay unbroken and last={
					\node[anchor=south east, outer sep=0pt] at (\linewidth-width,0) {
						\textcolor{main}{$\clubsuit$}
					};
				}
		},
	notstyle/.style={
			common,
			colframe=noteorange,
			colback=noteorange!5,
			colbacktitle=noteorange,
			overlay unbroken and last={
					\node[anchor=south east, outer sep=0pt] at (\linewidth-width,0) {
						\textcolor{noteorange}{$\spadesuit$}
					};
				}
		},
}

%自定义命令
\renewcommand\boxed[1]{\begin{framed}#1\end{framed}}
%多行公式对齐方框
\newcommand*\widefbox[1]{\fbox{\hspace{1em}#1\hspace{1em}}}
\definecolor{myblue}{rgb}{.8, .8, 1}
\newcommand*\bluebox[1]{%
	\colorbox{myblue}{\hspace{1em}#1\hspace{1em}}
}

%行内注意
\newenvironment{note}{
	\par\noindent\makebox[-3pt][r]{
		\scriptsize\color{notered}\textdbend\quad}
	\textbf{\color{noteorange}注意} \bfseries
}
{\par}

%例题环境
\newcounter{exam}[chapter]
\setcounter{exam}{0}
\renewcommand{\theexam}{\thechapter.\arabic{exam}}
\newenvironment{example}
{
	\refstepcounter{exam}
	\begin{tcolorbox}[exastyle,title=例题\,\theexam]
		{\bfseries 题目\,\,}\bfseries
		}
		{
	\end{tcolorbox}
}

%注意环境
%{
%	\begin{tcolorbox}[notstyle,title=笔记]
%		}
%		{
%	\end{tcolorbox}
%}
\newenvironment{notice}{\begin{tcolorbox}[breakable,enhanced,leftrule=3pt,
    rightrule=0pt,toprule=0pt,bottomrule=0pt,colback=noteorange!5,
    colframe=noteorange,sharp corners]
{\color{noteorange}\zihao{2}\faPencilSquareO \bfseries 笔记}\par
}{\end{tcolorbox}}

\newenvironment{summary}{
  \begin{tcolorbox}[
    breakable,enhanced,leftrule=5pt,rightrule=0pt,
    toprule=0pt,bottomrule=0pt,colback=blue!5,
    colframe=blue,sharp corners
    ]
    {\color{blue}\zihao{2}\faFolderO \bfseries 总结}\par
}
{
\end{tcolorbox}
}
